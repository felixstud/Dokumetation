%!TEX root = Uno-Dokumentation.tex
\chapter{Ausblick}
Das Projekt bietet einiges an Optimierungs- und Verbesserungspotential. Die Optik der Software ist noch auf keinem marktreifen Zustand und kann durch ein einheitliches Design, welches sich auf allen Seiten wiederfindet erheblich verbessert werden. Animationen, beispielsweise beim Ablegen einer Karte kann den Benutzern helfen die Spielzüge besser nachzuvollziehen und vereinfacht es dem Spielgeschehen zu folgen. Neben der optischen Seite, kann auch der Code optimiert werden. Überflüssige Klassenmember und Methoden können entfernt werden. Weitere Klassen können den Code besser verständlich und weniger komplex machen. Durch die objektorientierte Programmierung kann der Code einfach erweitert und optimiert werden. Zahlreiche Kommentare an nahezu jeder Methode, geben Entwicklern, die den Code zuvor noch nicht gesehen haben, die Möglichkeit ihn zu verstehen und weiterzuentwickeln. Die Client-Server-Verbindungen könnte optimiert werden, damit jedes Spiel zuverlässig funktioniert.\\
Die feste Anzahl von Spielern schränkt das Spielerlebnis erheblich ein. Mit einer Auswahl am Anfang des Spiels könnte der Spieler, der als Server agiert, festlegen mit wie viel Spielern das Spiel zu spielen ist. Dabei ist eine Anzahl laut der Regeln von zwei bis zehn Spielern sinnvoll \cite{Mattel}. Die Implementierung von Spezialkarten und einem UNO-Button, der signalisieren soll, dass ein Spieler nur noch eine Karte auf der Hand hat, würde das Spiel dem realen Spiel deutlich näher bringen. 