%!TEX root = Uno-Dokumentation.tex

\chapter{Bedingungen}
\label{ch:bedingungen}
Entgegen des Ziels dieser Arbeit, wird die Programmierung der Software durch eine einzelne Person durchgeführt. Damit das Projekt in der vorgegebenen Zeit für eine Einzelperson umsetzbar ist, ist es notwendig vorab Prioritäten zu definieren. An aller erster Stelle steht die Funktion der Software. Die Optischen Eigenschaften stehen an letzter Stelle. Es werden nicht alle Spielregeln implementiert, nur die, die für ein einfaches Spielerlebnis ausreichend sind.\\
Folgende Hauptanforderungen sind für die Umsetzung des Projektes definiert:
\begin{itemize}
	\item Spielbar mit vier Spielern über ein Netzwerk 
	\item Grundlegende Spiellogik
	\item Grafische Benutzeroberfläche / Spielfläche
	\item Objektorientierte Programmierung
	\item Strukturierte Vorgehensweise
\end{itemize}
Neben den Hauptanforderungen sind folgende Nebenanforderungen definiert:
\begin{itemize}
	\item Intuitive Menüführung
	\item Farbige Karten
	\item Spezialkarten implementieren
\end{itemize}