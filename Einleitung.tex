%!TEX root = Uno-Dokumentation.tex
\chapter{Einleitung}
Software wird durch die stetig wachsenden Möglichkeiten immer wichtiger und komplexer. Automatisierungen und Virtualisierungen werden immer mehr eingesetzt und bieten durch ihre hohe Komplexität eine echte Alternative zur Arbeit von Menschenhand. Auch die Videospielindustrie wächst immer weiter und erfreut sich immer mehr Nutzern. Auch hier bieten die Weiterentwicklungen von Programmiersprachen und Erweiterungen immer mehr Möglichkeiten. Den Nachteilen dieser ganzen neuen Möglichkeiten stehen die Programmierer entgegen. Software mit heutigen Standards können nur schwer bis gar nicht von einem einzelnen Softwareentwickler programmiert werden. Das Arbeiten im Team gehört schon lange dazu, marktreife Software zu produzieren. Die Arbeit im Team erfordert umfangreiche Planung, Organisation und Überwachung damit alle Entwickler zusammen an einem Projekt arbeiten können. Im Rahmen des Moduls \textit{Softwareengineering} soll ein Gesellschaftsspiel programmiert werden und dabei gelernt werden, wie man als Team an einem Software-Projekt arbeitet und dieses von der Idee bis zum Endprodukt umsetzt.