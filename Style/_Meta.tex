% Informationen ------------------------------------------------------------
% 	Definition von globalen Parametern, die im gesamten Dokument verwendet
% 	werden können (z.B auf dem Deckblatt etc.).
% --------------------------------------------------------------------------
\newcommand{\titel}{Uno}
\newcommand{\untertitel}{Software-Engineering Projekt}
\newcommand{\semester}{-}
\newcommand{\art}{Vorlage}
\newcommand{\autor}{Felix Bauer}
\newcommand{\matrikel}{695033}
%\newcommand{\studienbereich}{Elektrotechnik}
\newcommand{\erstgutachter}{}
%\newcommand{\zweitgutachter}{}
\newcommand{\jahr}{2022}

% Eigene Befehle
%\newcommand{\todo}[1]{\textbf{\textsc{\textcolor{red}{(TODO: #1)}}}}
\newcommand{\todo}[1]{\textcolor{red}{(TODO:#1)}}
% Autorennamen in small caps
\newcommand{\AutorZ}[1]{\textsc{#1}}
\newcommand{\Autor}[1]{\AutorZ{\citeauthor{#1}}}

% Befehle zur semantischen Auszeichnung von Text
\newcommand{\NeuerBegriff}[1]{\textbf{#1}}
\newcommand{\Fachbegriff}[1]{\textit{#1}}
\newcommand{\Prozess}[1]{\textit{#1}}
\newcommand{\Webservice}[1]{\textit{#1}}
\newcommand{\Eingabe}[1]{\texttt{#1}}
\newcommand{\Code}[1]{\texttt{#1}}
\newcommand{\Datei}[1]{\texttt{#1}}
\newcommand{\Datentyp}[1]{\textsf{#1}}
\newcommand{\XMLElement}[1]{\textsf{#1}}

% Abkürzungen
\newcommand{\vgl}[1]{\mbox{vgl.~#1}}
\newcommand{\ua}{\mbox{u.\,a.\ }}
\newcommand{\zB}{\mbox{z.\,B.\ }}
\newcommand{\siehe}[1]{\mbox{s.~#1}}
\newcommand{\nach}[1]{\mbox{nach~#1}}
\newcommand{\mathe}[1][black]{$\mathrm{#1}$}
\newcommand{\hervorheben}{\rowcolor[gray]{0.95}}

\newcommand{\afz}[1]{,,#1''}