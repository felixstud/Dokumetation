% Anpassung des Seitenlayouts ----------------------------------------------
% 	siehe Seitenstil.tex
% --------------------------------------------------------------------------
\usepackage[headsepline,automark]{scrlayer-scrpage}

\usepackage[table]{xcolor} 
\definecolor{colBackground}{rgb}{0.95, 0.95, 0.95}
\definecolor{colKeyword}{rgb}{0, 0, 0.9}
\definecolor{colComment}{rgb}{0, 0.6, 0}
\definecolor{colString}{rgb}{0.8, 0, 0}

\usepackage{listingsutf8}
\lstset{language=C++,
	basicstyle =\footnotesize\selectfont\ttfamily,
	backgroundcolor=\color{colBackground},
	keywordstyle=\color{colKeyword},
	stringstyle=\color{colString},
	showstringspaces=false,
	commentstyle=\color{colComment},
	breaklines=true,
	numbers=left,
	stepnumber=1,
	numbersep=10pt,
	tabsize=2,
	captionpos=b,
	extendedchars=true,
	frame=single,
	captionpos=t
}

\renewcommand{\lstlistingname}{Codeausschnitt} % Damit Codebeispiele mit Code betitelt werden.



% Anpassung an Landessprache -----------------------------------------------
% 	Verwendet globale Option german siehe \documentclass
% --------------------------------------------------------------------------
\usepackage[ngerman]{babel}
\usepackage{pifont}

\usepackage[utf8]{inputenc}
\usepackage{amsmath,amssymb,units}
\usepackage{esvect}
\usepackage{wrapfig,caption, placeins}
\captionsetup{
	format = plain,
	justification = centering,
	labelsep = newline,
	singlelinecheck = false,
	labelfont = bf,
	font = small
}
\usepackage{mathptmx,charter,helvet,courier}
\usepackage[printonlyused]{acronym}
\usepackage{wrapfig}

% Umlaute ------------------------------------------------------------------
% 		Umlaute/Sonderzeichen wie äöüß direkt im Quelltext verwenden (CodePage).
%		Erlaubt automatische Trennung von Worten mit Umlauten.
% --------------------------------------------------------------------------
\usepackage[utf8]{inputenc}
\usepackage[T1]{fontenc}
\usepackage{ae}			 	% "schöneres" ä
\usepackage{textcomp} 		% Euro-Zeichen etc.
\usepackage[german=quotes]{csquotes}



% Verwendung von Blindtext zur Layout gestaltung -----------------------------------------------------
\usepackage{blindtext}

% Grafiken -----------------------------------------------------------------
% 		Einbinden von Grafiken [draft oder final]
% 		Option [draft] bindet Bilder nicht ein - auch globale Option
% --------------------------------------------------------------------------
\usepackage{graphicx}
\graphicspath{{Bilder/}} 	% Dort liegen die Bilder des Dokuments
\usepackage{subcaption}

% Befehle aus AMSTeX für mathematische Symbole z.B. \boldsymbol \mathbb ----
\usepackage{amsmath,amsfonts}

% Für Index-Ausgabe; \printindex -------------------------------------------
\usepackage{makeidx}

% Einfache Definition der Zeilenabstände und Seitenränder etc. -------------
\usepackage{setspace}
\usepackage{geometry}


%% Zum Umfließen von Bildern -------------------------------------------------
\usepackage{floatflt}


% Lange URLs umbrechen etc. -------------------------------------------------
\usepackage{url}

% f�r lange Tabellen
\usepackage{longtable}
\usepackage{array}
\usepackage{ragged2e}
\usepackage{lscape}
\usepackage{colortbl} %farbige hinterlegung


% Spaltendefinition rechtsbündig mit definierter Breite ---------------------
\newcolumntype{w}[1]{>{\raggedleft\hspace{0pt}}p{#1}}

% Formatierung von Listen ändern
\usepackage{paralist}
% Standardeinstellungen:
\setdefaultleftmargin{2.5em}{2.2em}{1.87em}{1.7em}{1em}{1em}

%Zur Verwendung von BibLaTex
%\usepackage[style=numeric]{biblatex}
%\addbibresource{literatur}

% Zur Richtigen Verwendung von Einheiten
\usepackage[locale=DE]{siunitx}

\sisetup{
	mode = text,
	detect-family,
	detect-weight,  
	exponent-product = \cdot,
	number-unit-separator=\text{\,},
	output-decimal-marker={\text{,}},
	%math-rm=\mathsf,
	%text-rm=\sffamily,
	range-phrase = { - }	
}
\DeclareSIUnit{\var}{var}

\usepackage{xcolor}
\usepackage{caption}
\usepackage{float}
%\usepackage{subfig}


% pdf-Optionen  --------------------------------------------------------------
\usepackage[
bookmarks,
bookmarksopen=true,
pdftitle={\titel},
pdfauthor={\autor},
pdfcreator={\autor},
pdfsubject={\titel},
pdfkeywords={\titel},
colorlinks=true,
%linkcolor=red, % einfache interne Verknüpfungen
%anchorcolor=black,% Ankertext
%citecolor=blue, % Verweise auf Literaturverzeichniseinträge im Text
%filecolor=magenta, % Verkn�pfungen, die lokale Dateien �ffnen
%menucolor=red, % Acrobat-Men�punkte
%urlcolor=cyan, 
% f�r die Druckversion k�nnen die Farben ausgeschaltet werden:
linkcolor=black, % einfache interne Verkn�pfungen
anchorcolor=black,% Ankertext
citecolor=black, % Verweise auf Literaturverzeichniseintr�ge im Text
filecolor=black, % Verkn�pfungen, die lokale Dateien �ffnen
menucolor=black, % Acrobat-Men�punkte
urlcolor=black, 
%backref,
%pagebackref,
plainpages=false,% zur korrekten Erstellung der Bookmarks
pdfpagelabels,% zur korrekten Erstellung der Bookmarks
hypertexnames=false,% zur korrekten Erstellung der Bookmarks
linktoc=all%Sowohl Seitenzahl als auch Text als Link
]{hyperref}

\addto\extrasngerman{\def
\subsectionautorefname{Kapitel}
} %Definition "Kapitel" bei autoref von subsection