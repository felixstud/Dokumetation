%!TEX root = Uno-Dokumentation.tex

\chapter{Zusammenfassung}
Das Ziel des Projektes war es, ein im Netzwerk spielbares Mehrspieler Spiel UNO zu programmieren. Da das Projekt nur von einer Person, anstatt von angedachten sechs Personen bearbeitet werden konnte, mussten einige Einschränkungen in Betracht gezogen werden. Eine einfache Spiellogik, nur grundlegende Regeln und Karten des Spiels und eine einfache, zweitrangige Optik.\\
Die Anfangs gesetzten Hauptanforderungen sind im Verlauf des Projektes erfüllt worden. Die Nebenanforderungen wurden außerdem erfüllt bis auf die Implementierung der Spezialkarten. Resultat des Projektes ist ein mit vier Spielern spielbares Spiel. Es kann im Netzwerk von unterschiedlichen Rechnern aus zusammen gespielt werden. Grundlegende Spiellogik für einfache Karten, ausgenommen von Spezialkarten, lassen erkennen, dass es sich bei dem Spiel um UNO handelt. Die Objektorientierte und strukturierte Programmierung der Software lässt andere Entwickler schneller den Code verstehen und gibt die Möglichkeit für Erweiterungen und Optimierungen. Eine einfache, aber intuitive Menüführung der gesamten Software bedarf keiner Erklärung für den Benutzer. Optisch bietet sich noch viel Optimierungsbedarf, wobei die Kernelemente für den Benutzer verständlich angeordnet und beispielsweise die Karten farbig sind. Der Fokus der Programmierung lag jedoch von Anfang an auf der Funktion und nicht auf der Optik.\\
Durch die Verwendung verschiedenster Tools für die strukturierte Programmierung eines solchen Projekts konnte viel gelernt werden. Über Planung, Umsetzung, Versionsverwaltung und die Programmierung selbst. Das Projekt hat einen Einblick in die Arbeitsweise von Software-Entwicklern gegeben. Leider ist der Einblick in das Arbeiten im Team aufgrund von mangelnden Team-Mitgliedern zu kurz gekommen. Eine ungefähre Vorstellung konnte trotzdem erlangt werden.